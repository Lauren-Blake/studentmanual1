\chapter[Vocabulary Words]{Vocabulary Words and Roots to Know}
\section{Vocabulary Words to Know}

\begin{enumerate}
\item abbreviate - (v) to shorten, abridge
\item abstinence - (n) the act of refraining from pleasurable activity, e.g., eating or drinking
\item adulation - (n) high praise
\item adversity - (n) misfortune, an unfavorable turn of events
\item aesthetic - (adj) pertaining to beauty or the arts
\item amicable - (adj) friendly, agreeable
\item anachronistic - (adj) out-of-date, not attributed to the correct historical period
\item anecdote - (n) short, usually funny account of an event
\item anonymous - (adj) nameless, without a disclosed identity
\item antagonist - (n) foe, opponent, adversary

\bigskip
\textbf{Tone Vocabulary} - Tone is the attitude of the speaker or narrator.  It differs from mood, which describes what the reader feels.  Mood is the feeling or atmosphere that a piece of writing creates within the reader.  

\bigskip
\item accusatory- charging of wrong doing
\item apathetic- indifferent due to lack of energy or concern
\item awe- solemn wonder
\item bitter- exhibiting strong animosity as a result of pain or grief
\item cynical- questions the basic sincerity and goodness of people

\bigskip
\item arid - (adj) extremely dry or deathly boring
\item assiduous - (adj) persistent, hard-working
\item asylum - (n) sanctuary, shelter, place of refuge
\item benevolent - (adj) friendly and helpful
\item camaraderie - (n) trust, sociability amongst friends
\item censure - (v) to criticize harshly
\item circuitous - (adj) indirect, taking the longest route
\item clairvoyant - (adj) exceptionally insightful, able to foresee the future
\item collaborate - (v) to cooperate, work together
\item compassion - (n) sympathy, helpfulness or mercy

\bigskip
\textbf{Tone Vocabulary} - Tone is the attitude of the speaker or narrator.  It differs from mood, which describes what the reader feels.  Mood is the feeling or atmosphere that a piece of writing creates within the reader.  

\item condescension; condescending-a feeling of superiority
\item callous-unfeeling, insensitive to feelings of others
\item contemplative-studying, thinking, reflecting on an issue
\item critical-finding fault
\item choleric-hot-tempered, easily angered 

\bigskip
\item compromise - (v) to settle a dispute by terms agreeable to both sides
\item condescending - (adj) possessing an attitude of superiority, patronizing
\item conditional - (adj) depending on a condition, e.g., in a contract
\item conformist - (n) person who complies with accepted rules and customs
\item congregation - (n) a crowd of people, an assembly
\item convergence - (n) the state of separate elements joining or coming together
\item deleterious - (adj) harmful, destructive, detrimental
\item demagogue - (n) leader, rabble-rouser, usually appealing to emotion or prejudice
\item digression - (n) the act of turning aside, straying from the main point, esp. in a speech or argument
\item diligent - (adj) careful and hard-working

\bigskip
\textbf{Tone Vocabulary} - Tone is the attitude of the speaker or narrator.  It differs from mood, which describes what the reader feels.  Mood is the feeling or atmosphere that a piece of writing creates within the reader.  


\item contemptuous-showing or feeling that something is worthless or lacks respect
\item caustic-intense use of sarcasm; stinging, biting
\item conventional-lacking spontaneity, originality, and individuality
\item disdainful-scornful
\item didactic-author attempts to educate or instruct the reader 

\item discredit - (v) to harm the reputation of, dishonor or disgrace
\item disdain - (v) to regard with scorn or contempt
\item divergent - (adj) separating, moving in different directions from a particular point
\item empathy - (n) identification with the feelings of others
\item emulate - (v) to imitate, follow an example
\item enervating - (adj) weakening, tiring
\item enhance - (v) to improve, bring to a greater level of intensity
\item ephemeral - (adj) momentary, transient, fleeting
\item evanescent - (adj) quickly fading, short-lived, esp. an image
\item exasperation - (n) irritation, frustration

\bigskip
\textbf{Tone Vocabulary} - Tone is the attitude of the speaker or narrator.  It differs from mood, which describes what the reader feels.  Mood is the feeling or atmosphere that a piece of writing creates within the reader.  


\item derisive-ridiculing, mocking
\item earnest-intense, a sincere state of mind
\item erudite-learned, polished, scholarly
\item fanciful-using the imagination
\item forthright- directly frank without hesitation 

\bigskip
\item exemplary - (adj) outstanding, an example to others
\item extenuating - (adj) excusing, lessening the seriousness of guilt or crime, e.g., of mitigating factors
\item florid - (adj) red-colored, flushed; gaudy, ornate
\item fortuitous - (adj) happening by luck, fortunate
\item frugal - (adj) thrifty, cheap
\item hackneyed - (adj)  lichéd, worn out by overuse
\item haughty - (adj) arrogant and condescending
\item hedonist - (n) person who pursues pleasure as a goal
\item hypothesis - (n) assumption, theory requiring proof
\item impetuous - (adj) rash, impulsive, acting without thinking

\bigskip
\textbf{Tone Vocabulary} - Tone is the attitude of the speaker or narrator.  It differs from mood, which describes what the reader feels. Mood is the feeling or atmosphere that a piece of writing creates within the reader.  


\item gloomy-darkness, sadness, rejection
\item haughty-proud and vain to the point of arrogance
\item indignant-marked by anger aroused by injustice
\item intimate-very familiar
\item judgmental-authoritative and often having critical opinions 

\bigskip
\item impute - (v) to attribute an action to particular person or group
\item incompatible - (adj) opposed in nature, not able to live or work together
\item inconsequential - (adj) unimportant, trivial
\item inevitable - (adj) certain, unavoidable
\item integrity - (n) decency, honesty, wholeness
\item intrepid - (adj) fearless, adventurous
\item intuitive - (adj) instinctive, untaught
\item jubilation - (n) joy, celebration, exultation
\item lobbyist - (n) person who seeks to influence political events
\item longevity - (n) long life

\bigskip
\textbf{Tone Vocabulary}-Tone is the attitude of the speaker or narrator.  It differs from mood, which describes what the reader feels.  Mood is the feeling or atmosphere that a piece of writing creates within the reader.  

\bigskip
\item jovial-happy
\item lyrical-expressing a poet's inner feelings; emotional; full of images; song-like
\item matter-of-fact-accepting of conditions; not fanciful or emotional
\item mocking-treating with contempt or ridicule
\item morose-gloomy, sullen, surly, despondent 

\bigskip
\item mundane - (adj) ordinary, commonplace
\item nonchalant - (adj) calm, casual, seeming unexcited
\item novice - (n) apprentice, beginner
\item opulent - (adj) wealthy
\item orator – (n) lecturer, speaker
\item ostentatious - (adj) showy, displaying wealth
\item parched - (adj) dried up, shriveled
\item perfidious - (adj) faithless, disloyal, untrustworthy
\item precocious - (adj) unusually advanced or talented at an early age
\item pretentious - (adj) pretending to be important, intelligent or cultured

\bigskip
\textbf{Tone Vocabulary} - Tone is the attitude of the speaker or narrator.  It differs from mood, which describes what the reader feels.  Mood is the feeling or atmosphere that a piece of writing creates within the reader.  

\bigskip
\item malicious-purposely hurtful
\item objective-an unbiased view-able to leave personal judgments aside
\item optimistic-hopeful, cheerful
\item obsequious-polite and obedient in order to gain something
\item patronizing-air of condescension 

\bigskip
\item procrastinate - (v) to unnecessarily delay, postpone, put off
\item prosaic - (adj) relating to prose; dull, commonplace
\item prosperity - (n) wealth or success
\item provocative - (adj) tending to provoke a response, e.g., anger or disagreement
\item prudent - (adj) careful, cautious
\item querulous - (adj) complaining, irritable
\item rancorous - (adj) bitter, hateful
\item reclusive – (adj) preferring to live in isolation
\item reconciliation - (n) the act of agreement after a quarrel, the resolution of a dispute
\item renovation - (n) repair, making something new again

\bigskip
\textbf{Tone Vocabulary} - Tone is the attitude of the speaker or narrator.  It differs from mood, which describes what the reader feels.  Mood is the feeling or atmosphere that a piece of writing creates within the reader.  

\bigskip
\item pessimistic-seeing the worst side of things; no hope
\item quizzical-odd, eccentric, amusing
\item ribald-offensive in speech or gesture
\item reverent-treating a subject with honor and respect
\item ridiculing-slightly contemptuous banter; making fun of

\bigskip

\item resilient - (adj) quick to recover, bounce back
\item restrained - (adj) controlled, repressed, restricted
\item reverence - (n) worship, profound respect
\item sagacity - (n) wisdom
\item scrutinize - (v) to observe carefully
\item spontaneity - (n) impulsive action, unplanned events
\item spurious - (adj) lacking authenticity, false
\item submissive -  (adj) tending to meekness, to submit to the will of others
\item substantiate - (v) to verify, confirm, provide supporting evidence
\item subtle - (adj) hard to detect or describe; perceptive

\bigskip
\textbf{Tone Vocabulary} - Tone is the attitude of the speaker or narrator.  It differs from mood, which describes what the reader feels.  Mood is the feeling or atmosphere that a piece of writing creates within the reader.  

\bigskip
\item reflective-illustrating innermost thoughts and emotions
\item sarcastic-sneering, caustic
\item sardonic-scornfully and bitterly sarcastic
\item satiric-ridiculing to show weakness in order to make a point, teach
\item sincere-without deceit or pretense; genuine
 
\bigskip
\item superficial - (adj) shallow, lacking in depth
\item superfluous - (adj) extra, more than enough, redundant
\item suppress - (v) to end an activity, e.g., to prevent the dissemination of information
\item surreptitious - (adj) secret, stealthy
\item tactful - (adj) considerate, skillful in acting to avoid offense to others
\item tenacious - (adj) determined, keeping a firm grip on
\item transient - (adj) temporary, short-lived, fleeting
\item venerable - (adj) respected because of age
\item vindicate - (v) to clear from blame or suspicion
\item wary - (adj) careful, cautious

\bigskip
Tone Vocabulary - Tone is the attitude of the speaker or narrator.  It differs from mood, which describes what the reader feels.  Mood is the feeling or atmosphere that a piece of writing creates within the reader.  

\bigskip
\item solemn-deeply earnest, tending toward sad reflection
\item sanguineous - optimistic, cheerful
\item whimsical- odd, strange, fantastic; fun
\end{enumerate}

\section{Latin and Greek Roots to Know}

\begin{spacing}{2}
\begin{enumerate}
\item Acer-, acid-, acri- means sharp. Based on this definition, what is an acerbic food or drink?
\item Ag-, agi-, ig-, act- means do, move, or go. What do you think that the definition of agitate means?
\item Arch means chief, first, rule means. What do you think that archaic means?
\item Belli- means war. What do you think that bellicose means?
\item Carp-, cip-, cept means to take. What do you think that reciprocate means?
\item Ced-, ceed-, cede-, cess- means move, yield, go, surrender. What do you think that concede means?
\item Dict- means to say or speak. What do you think that contradict means?
\item Duc-, duct- means to lead. What do you think that induce means?
\item Fac-, fact-, fic-, fect- means to do or make. What do you think that infect means? 
\item Fall-, fals- means to deceive. What do you think that fallacious means?
\item Fid-, fide-, feder- means faith or trust. What do you think that infidelity means?
\item Grad-, gress- means step or go. What do you think that egress means?
\item Greg- means herd. What do you think that gregarious means?
\item Homo- means same whereas hetero- means different. What do you think that homozygous means? What do you think that heterozygous means?
\item Ject- means insert. What do you think that eject means?
\item Loqu-, locut- means to talk or to speak. What do you think loquacious means?
\item Magn- means great. What do you think magnanimity means?
\item Migra- means wander. What do you think that migration means?
\item Neo- means new. What do you think that neophyte means?
\item Oligo- means few or little. What do you think that oligarchy means?
\item Pel-, puls- means drive or urge. What do you think repulsion means?
\item Pon-, pos-, pound- means place or put. What do you think that proponent means?
\item Reg- or recti- means straighten. What do you think that rectify means?
\item Sacr-, sanc-, secr- means sacred. What do you think that desecrate means?
\item Sat-, satis- means enough. What do you think satiate means?
\item Sed-, sess-, sid means sit. What do you think that subside means?
\item Solv, solu- means loosen. What do you think that absolve means?
\item Ten-, tin-, tain- means hold. What do you think that untenable means?
\item Tract-, tra- means draw or pull. What do you think that tractable means?
\item Vac- means empty. What do you think evacuate means?
\item Ven-, vent- means come. What do you think that intervene means?
\item Viv-, vita-, vivi- means alive or life. What do you think that vivacious means?
\item Vor- means to eat greedily. What do you think voracious means?
\end{enumerate}
\end{spacing}