\chapter[Reading Comprehension]{Reading Comprehension Part I}
\section{SAT worksheet 1F: Warm-up}
\textit{Directions: Turn your textbook to a passage-based reading section. Use this to help you answer the questions below.}

\begin{enumerate}
\item What is the structure of the types of passages seen in the critical reading section? (e.g. short passages)
\vfill\item What types of questions are being asked about the passages?
\vfill\item Introduction to critical reading strategy \#1: Look at a long passage. What do you notice about the line numbers of the questions following the long passage?
\end{enumerate}

\vfill
\newpage
\section[Reading Questions]{Reading Comprehension and Passage-Based Reading\\ Questions}
Besides Sentence Completions, the Critical Reading section of the SATs is composed of passage-based reading questions.  The passages can range from about 100 to 850 words.  They are drawn from a wide range of sources, including natural sciences, literary fiction, and social studies.  

\bigskip
Critical Reading questions test your understanding of the written word and your ability to read carefully and analytically.  They also test your vocabulary.  Some questions are based on a single passage, while other questions ask you to compare and contrast two related questions, usually based around the same topic or theme.

\bigskip
When reading the passages, it is important to read actively.  Keep in mind that you will be answering questions about the purpose and tone of the passage.  Take note of the style and content—is it fact-based or opinion-based?  Some students benefit from skimming the questions before reading the passage so you know what to look for. Reading as much as you can is the best way to improve your reading comprehension skills and expand your vocabulary.

\bigskip
The questions about the passages will generally go in order that they appear in the passage. There are four main types of questions:

\begin{itemize}
\item Main idea
\item Details
\item Inferences
\item Tone
\end{itemize}

\bigskip
\textbf{Main idea questions} generally ask you about the purpose or central theme of a passage.

\bigskip
\textbf{Detail questions} usually refer you to a specific line and ask you to explain a phrase or define a vocabulary word in context.

\bigskip
\textbf{Inference questions} require your analytical skills to come into play.  Based on the reading, what conclusions can you draw?  Make sure these conclusions could be supported with evidence directly from the reading.

\bigskip
\textbf{Tone questions} ask about the narrator's attitude.  Try to hear the author's voice in your head.  If you were reading the excerpt out loud, how would it sound? 

\bigskip
\textit{Keep in mind that the answers come directly from the passages.  Even if you have pre-existing knowledge about a subject, you must answer based on the reading! Therefore, it is important to always go back to the passage to find evidence for the correct answer. This is EXTREMELY important and this will come up over and over again in this chapter.}

\newpage
\section[Reading the Passage]{Strategies for Reading the Passage}

\underline{Method \#1: Read the entire passage}

Some people will read entire passage (they may or may not read the questions first), underlining and making short notes on passage and then answer the questions. 

\bigskip
\begin{enumerate}
\item What do you think are the benefits to this method?
\vfill\item What are the drawbacks to this method?
\end{enumerate}


\vfill
\underline{Method \#2: Back and Forth Method}

In the passage-based reading section, the questions are arranged by the order in which they refer to the passage not by their level of difficulty. Therefore, you can read the first question, see what the line number is in this question and read the beginning of the passage to two line numbers after the question. Then, answer the question. See where the line number in the next question is and then read from where you left off to slightly after the line number of the question. Read to there and then answer the question. The evidence for the correct answer in questions without line numbers can usually be found in between the line numbers listed in the question before and the question after it. 

\begin{enumerate}
\item Repeat reading the question, reading a part of the text, and answering the question until all you have left are questions about the passage as a whole. Complete these at the end when you have finished the passage. 

\vfill
\begin{enumerate}
\item What do you think are the benefits to this method?
\vfill\item What are the drawbacks to this method?
\end{enumerate}
\end{enumerate}

\vfill
\newpage

\section[Anticipating the Passage]{SAT Worksheet Practice 2F: Anticipating the Passage}
\textit{Directions: Read the following italicized passages that would appear before a passage. Then, answer the questions that follow.}

\bigskip
\textit{The following passage is from a neurosurgeon's autobiography recalling a discussion on alternative medicine.}\\
\begin{enumerate}
\item What do you think that the doctor's perspective will be throughout the passage? \hrulefill  \\
\item What point of view, if expressed in the passage, would suprise you? Why? \hrulefill \\
\end{enumerate}

\bigskip
\textit{The following passage is from an article entitled, ``Is your child busy all day?: a manifesto for unstructured play''}\\
\begin{enumerate}
\item What topic do you think will be discussed in the article? \hrulefill
\item What perspective to you think the author will take on this topic? \hrulefill
\end{enumerate}

\newpage
\section[Passage-Based Section]{Strategies for Answering the Passage-Based Reading Section}

\begin{enumerate}[leftmargin=0cm,labelwidth=\itemindent,labelsep=0cm,align=left,label={\bfseries Strategy \#\arabic*:\ }]
\item Do not use your own \longline or

\longline to answer the questions. You must only use the information presented in \longline. (This can be harder than it seems!)
\item Use \longline from the passage in order to select the correct answer.

\bigskip
\item Good hints for the correct answer can often be found around the line numbers give for a question. Therefore, if you are reading a question with line number, make sure to carefully read a little \longline and a little \longline each question.  

\bigskip
\item Before reading the answer choices, think about your \longline to the question. This will help you to not get distracted by trick answers or answers that are “sort of” right. 

\bigskip
\item For each question, you are looking for the \longline possible answer. Sometimes, an answer choice might be ``sort of" right. It can be tempting to pick these types of answer choices, but usually there is a better answer choice. The correct answer choice must have evidence from the text, so you should not have to ``stretch" what you think is the correct answer.

\bigskip
\item Also remember that the SAT is trying to trick you by writing answer choices with information stated in the passage but that is \longline to the question (see main reasons why SAT passage-based answer choices are wrong, later in this section). If you are still having trouble, \longline it for now and come back to it after you have answered the general questions about the passage. 

\bigskip
\item If you are having trouble finding the correct answer, eliminating

\longline answers or asking yourself ``is this answer wrong and why" for each answer choice can be easier than finding the \longline answer right away.

\bigskip
\item After you are finished with the passage, you will want to think about the

\longline and other elements of the passage and the feelings conveyed in it. This will help you to answer the questions, particularly about the passage overall or the tone of the passage. 

\centerline{\textbf{Think about an APE saying O!}}

\textbf{About.} What is the passage about? What is the main point or argument? 

\bigskip
\textbf{Purpose.} Why is the author writing this text? What the purpose?

\bigskip
\textbf{Expressing Attitude.} How does the author's attitude towards the topic relate to the main idea or point being made? How does the author use language, sentence structure, and rhetorical devices, such as similes and metaphors to express his/her attitude towards the topic

\bigskip
\textbf{Overall feeling (mood).} What does the reader feel towards the topic after reading the passage? 

\bigskip
\item When you get a question wrong, understand why the answer you originally chose is wrong in addition to finding evidence from the passage for the line number. To help you, we have a list of five reasons why incorrect answer choices are wrong on the next page. 
\end{enumerate}

\newpage

\section[Main Ideas and Details]{SAT Worksheet 3F: Practice with Main Ideas and Detail Questions}
\textit{Directions: Read the following passages. Write the main idea of the passage \textbf{in your own words.} Then, use context clues to determine the meaning of the bolded word or phrase \textbf{in the context of the passage.} and answer the other questions if applicable.}

\bigskip
\begin{linenumbers*}
\modulolinenumbers[5]
\indent The architect should be equipped with knowledge of many branches of study and varied kinds of learning, for it is by his judgement that all work done by the other arts is put to test. This knowledge is the child of practice and theory. Practice is the continuous and regular \textbf{exercise} of employment where manual work is done with any necessary material according to the design of a drawing. Theory, on the other hand, is the ability to demonstrate and explain the productions of dexterity on the principles of proportion.
\end{linenumbers*}

\begin{enumerate}
\item Main idea: \hrulefill
\item What is the meaning of ``exercise'': \hrulefill
\item What is the ``child of practice and theory''? \hrulefill
\end{enumerate}

\textit{This passage is from The Education of an Architect by Vitruvius and was obtained from The Project Gutenberg.}

\bigskip
\begin{linenumbers*}
\modulolinenumbers[5]
\indent Candide, all stupefied, could not yet very well realise how he was a hero. He resolved one fine day in spring to go for a walk, marching straight before him, believing that it was a privilege of the human as well as of the animal species to make use of their legs as they pleased. He had advanced two leagues when he was overtaken by four others, heroes of six feet, who bound him and carried him to a dungeon. He was asked which he would like the best, to be whipped six-and-thirty times through all the regiment, or to receive at once twelve balls of lead in his brain. He vainly said that human will is free, and that he chose neither the one nor the other. He was forced to make a choice; he determined, in virtue of that gift of God called liberty, to run the gauntlet six-and-thirty times. He bore this twice. The regiment was composed of two thousand men; that composed for him four thousand strokes, which laid bare all his muscles and nerves, from the \textbf{nape} of his neck quite down to his rump. As they were going to proceed to a third whipping, Candide, able to bear no more, begged as a favour that they would be so good as to shoot him. He obtained this favour; they bandaged his eyes, and bade him kneel down. The King of the Bulgarians passed at this moment and ascertained the nature of the crime. As he had great talent, he understood from all that he learnt of Candide that he was a young metaphysician, extremely ignorant of the things of this world, and he accorded him his pardon with a clemency which will bring him praise in all the journals, and throughout all ages.
\end{linenumbers*}

\begin{enumerate}
\item Main Idea: \hrulefill
\item Meaning of ``nape'': \hrulefill
\item What was Candide's decision? \hrulefill
\end{enumerate}

\textit{This passage is from Candide by  Philip Littell and was obtained from The Project Gutenberg.}

\hrulefill

\bigskip
\begin{linenumbers*}
\modulolinenumbers[5]
\indent In this manner, the mysterious old Roger Chillingworth became the medical adviser of the Reverend Mr. Dimmesdale. As not only the disease interested the physician, but he was strongly moved to look into the character and qualities of the patient, these two men, so different in age, came gradually to spend much time together. For the sake of the minister's health, and to enable the leech to gather plants with healing balm in them, they took long walks on the sea-shore, or in the forest; mingling various walks with the splash and murmur of the waves, and the solemn wind-anthem among the tree-tops. Often, likewise, one was the guest of the other in his place of study and retirement. There was a fascination for the minister in the company of the man of science, in whom he recognised an intellectual cultivation of no moderate depth or scope; together with a range and freedom of ideas, that he would have \textbf{vainly} looked for among the members of his own profession. In truth, he was startled, if not shocked, to find this attribute in the physician.
\end{linenumbers*}

\begin{enumerate}
\item Main idea: \hrulefill
\item Meaning of ``vainly'': \hrulefill
\item What activities would the two men do together? \hrulefill
\end{enumerate}

\begin{linenumbers}
\modulolinenumbers[5]
\indent Mr. Dimmesdale was a true priest, a true religionist, with the reverential sentiment largely developed, and an order of mind that impelled itself powerfully along the track of a creed, and wore its passage continually deeper with the lapse of time. In no state of society would he have been what is called a man of liberal views; it would always be essential to his peace to feel the pressure of a faith about him, supporting, while it confined him within its iron framework. Not the less, however, though with a tremulous enjoyment, did he feel the occasional relief of looking at the universe through the \textbf{medium} of another kind of intellect than those with which he habitually held converse. It was as if a window were thrown open, admitting a freer atmosphere into the close and stifled study, where his life was wasting itself away, amid lamp-light, or obstructed day-beams, and the musty fragrance, be it sensual or moral, that exhales from books. But the air was too fresh and chill to be long breathed with comfort. So the minister, and the physician with him, withdrew again within the limits of what their Church defined as orthodox.
\end{linenumbers}

\begin{enumerate}
\item Main Idea: \hrulefill
\item Meaning of ``medium'': \hrulefill
\item In your own words, what is the thing that ``exhales from books''? \hrulefill
\end{enumerate}

\begin{linenumbers}
\modulolinenumbers[5]
\indent Thus Roger Chillingworth scrutinised his patient carefully, both as he saw him in his ordinary life, keeping an accustomed pathway in the range of thoughts familiar to him, and as he appeared when thrown amidst other moral scenery, the novelty of which might call out something new to the surface of his character. He deemed it essential, it would seem, to know the man, before attempting to do him good. Wherever there is a heart and an intellect, \textbf{the diseases of the physical frame are tinged with the peculiarities of these.} In Arthur Dimmesdale, thought and imagination were so active, and sensibility so intense, that the bodily infirmity would be likely to have its groundwork there. So Roger Chillingworth--the man of skill, the kind and friendly physician--strove to go deep into his patient's bosom, delving among his principles, prying into his recollections, and probing everything with a cautious touch, like a treasure-seeker in a dark cavern.
\end{linenumbers}

\begin{enumerate}
\item Main Idea: \hrulefill
\item Meaning of the bolded phrase: \hrulefill
\end{enumerate}

\begin{linenumbers}
\modulolinenumbers[5]
\indent Few secrets can escape an investigator, who has opportunity and licence to undertake such a quest, and skill to follow it up. A man burdened with a secret should especially avoid the intimacy of his physician. If the latter possess native \textbf{sagacity,} and a nameless something more,—let us call it intuition; if he show no intrusive egotism, nor disagreeable prominent characteristics of his own; if he have the power, which must be born with him, to bring his mind into such affinity with his patient's, that this last shall unawares have spoken what he imagines himself only to have thought; if such revelations be received without tumult, and acknowledged not so often by an uttered sympathy as by silence, an inarticulate breath, and here and there a word to indicate that all is understood; if to these qualifications of a confidant be joined the advantages afforded by his recognised character as a physician;—then, at some inevitable moment, will the soul of the sufferer be dissolved, and flow forth in a dark but transparent stream, bringing all its mysteries into the daylight.
\end{linenumbers}

\begin{enumerate} 
\item Main idea: \hrulefill
\item Meaning of sagacity: \hrulefill 
\end{enumerate}

\begin{linenumbers}
\modulolinenumbers[5]
\indent Roger Chillingworth possessed all, or most, of the attributes above \textbf{enumerated.} Nevertheless, time went on; a kind of intimacy, as we have said, grew up between these two cultivated minds, which had as wide a field as the whole sphere of human thought and study to meet upon; they discussed every topic of ethics and religion, of public affairs, and private character; they talked much, on both sides, of matters that seemed personal to themselves; and yet no secret, such as the physician fancied must exist there, ever stole out of the minister's consciousness into his companion's ear. The latter had his suspicions, indeed, that even the nature of Mr. Dimmesdale's bodily disease had never fairly been revealed to him. It was a strange reserve!
\end{linenumbers}

\begin{enumerate}
\item Main Idea: \hrulefill
\item Meaning of ``enumerated": \hrulefill 
\item What is the relationship between Chillingworth and Dimmesdale? \hrulefill
\end{enumerate}

\bigskip
\textbf{Now, identify the main idea of the entire passage (the last 5 paragraphs).}

\bigskip
\hrulefill

\bigskip
\textit{This passage is from The Scarlet Letter by Nathaniel Hawthorne and obtained from The Project Gutenberg.}

\bigskip
\textbf{\textit{This exercise demonstrates how you can break down a long passage by paragraphs in order to better grasp the meaning of the entire passage. }}

\newpage
\section[Tone and Inference]{Strategies for Answering Tone and Inference Questions}
You should be engaged with the text as you read, asking questions, absorbing information, and determining its purpose.
Inference questions ask you to draw conclusions and make evaluations and inferences.  These kinds of questions require you to go beyond reading and understanding the text—they ask you to interpret and come to a conclusion through reasoning.

\bigskip
Words like probably, apparently, seems, suggests, it can be inferred, and the author implies usually indicate that it is an extended reasoning question. When there are two passages, both are based on a shared issue or theme.  The two authors either contradict, support, or complement each other's point of view. Usually Two Passage questions will ask you to interpret point of view, tone, style, or attitude.

\bigskip
The italicized headnote at the beginning may help you make a prediction as to tone and style.  If the introduction says the excerpt was adapted from a scientific article, you may expect that the tone is dry, neutral, informational, clinical, etc.  If it is excerpted from a book by Mark Twain, you can guess that it may be humorous, mocking, satirical, witty, and wordy.

\bigskip
Tone can usually be determined by looking at the specific details and words of a passage. What adjectives are used to describe a person or verbs explain how he/she moves or talks?

\bigskip
Is the wording formal or informal, elevated or lowbrow, academic or colloquial?  Notice any details that jump out as odd—for example, an author refers to a character as a wolf-why would (s)he do that?

\bigskip
Try to hear the author's voice in your head.  If you were reading the excerpt out loud, how would it sound? 

\bigskip
Usually with tone questions, you will be able to easily eliminate one or more choices because they are way off.  Generally, the writers are moderate in their opinions.  Extreme tones like outraged, despairing, or overjoyed usually are not correct. See the tone words commonly used on the SATs in the last chapter of this document.

\newpage
\section[Inference and Tone Practice]{SAT Worksheet 3F: Practice with Inference and Tone Questions}
\textit{Directions: Read the following passages and complete the questions that follow}

\bigskip
\textit{From O. Henry's The Gift of the Magi}

\bigskip
\begin{linenumbers*}
\modulolinenumbers[5]
\indent ONE dollar and eighty-seven cents. That was all. And sixty cents of it was in pennies. Pennies saved one and two at a time by bulldozing the grocer and the vegetable man and the butcher until one's cheeks burned with the silent imputation of parsimony that such close dealing implied. Three times Della counted it. One dollar and eighty-seven cents. And the next day would be Christmas.

\indent There was clearly nothing to do but flop down on the shabby little couch and howl. So Della did it. Which instigates the moral reflection that life is made up of sobs, sniffles, and smiles, with sniffles predominating.
\end{linenumbers*}

\begin{enumerate}
\item Why is Della crying?
\vfill\item What does Christmas have to do with her crying?  
\vfill\item Does her life seem generally happy or sad?  
\vfill\item How would you describe the style and/or tone?
\end{enumerate}

\vfill

\newpage
\textit{From Ray Bradbury's All Summer in a Day}

\bigskip
\begin{linenumbers*}
\modulolinenumbers[5]
\indent They turned on themselves, like a feverish wheel, all tumbling spokes. Margot stood alone. She was a very frail girl who looked as if she had been lost in the rain for years and the rain had washed out the blue from her eyes and the red from her mouth and the yellow from her hair. She was an old photograph dusted from an album, whitened away, and if she spoke at all her voice would be a ghost. Now she stood, separate, staring at the rain and the loud wet world beyond the huge glass.
``What're you looking at ?" said William.

Margot said nothing.

``Speak when you're spoken to."

\indent He gave her a shove. But she did not move; rather she let herself be moved only by him and nothing else. They edged away from her, they would not look at her. She felt them go away. And this was because she would play no games with them in the echoing tunnels of the underground city. If they tagged her and ran, she stood blinking after them and did not follow. When the class sang songs about happiness and life and games her lips barely moved. Only when they sang about the sun and the summer did her lips move as she watched the drenched windows.

\indent And then, of course, the biggest crime of all was that she had come here only five years ago from Earth, and she remembered the sun and the way the sun was and the sky was when she was four in Ohio. And they, they had been on Venus all their lives, and they had been only two years old when last the sun came out and had long since forgotten the color and heat of it and the way it really was.

\indent But Margot remembered.

\indent ``It's like a penny," she said once, eyes closed.

\indent ``No it's not!" the children cried.

\indent ``It's like a fire," she said, ``in the stove."

\indent ``You're lying, you don't remember!" cried the children.

\indent But she remembered and stood quietly apart from all of them and watched the patterning windows. 
\end{linenumbers*}

\begin{enumerate}
\item How would you describe Margot's state of mind?  What imagery is used to reveal this?
\vfill\item Why do you think she feels that way?
\vfill\item How do the other kids feel about Margot?  Why?
\vfill\item How would you describe the style and/or tone?
\end{enumerate}

\vfill
\section[Strategy]{Strategy for Analyzing Incorrect Answers}
While it can be frustrating to get a question wrong, it can be helpful to see why the answer that you selected was incorrect. Furthermore, understanding why the incorrect answers are incorrect can also alert you to the answer choices that the SAT question writers will use to try get you to select the incorrect answers. 

\bigskip
\textbf{Why Incorrect SAT Answer Choices on the Reading Sections are Wrong}

\begin{enumerate}
\item \textbf{Too Broad/Require too much of a leap:} Sometimes the passage is about a specific example (like a mammal) and then the answer choice will be \longline (like about animals). The answer choice might feel right or that the statement may be something the author would agree with, but it is usually not the best answer. 

\item \textbf{Too narrow:} This type of choice might be

\longline in the text, but doesn't completely answer the question. This is common in questions about the \longline of the passage.

\item \textbf{Too Extreme:} The SATs want to test how carefully you can read and

\longline a text. Therefore, words like \longline,

\longline, \longline, \longline, and \longline are probably not good choices. 


\item \textbf{Not Stated in the Passage:} The answer choice is not \longline anywhere in the passage or is \longline to the passage. This type of answer choice could make sense, but is not close enough to what is said in the passage. 

\item \textbf{True but unrelated to the Question:} This type of answer choice might be

\longline and \longline but it doesn't

\longline the question. 
\end{enumerate}

\newpage
\section[Passage-Based Reading]{SAT Worksheet 4F: Passage-Based Reading Practice Questions}
\textit{Read the following passages and complete the questions that follow.}

\bigskip
\textit{From Kate Chopin's 1899 novel, The Awakening}

\bigskip
\begin{linenumbers*}
\modulolinenumbers[5]
\indent Her marriage to Leonce Pontellier was purely an accident, in this respect resembling many other marriages which masquerade as the decrees of Fate. It was in the midst of her secret great passion that she met him. He fell in love, as men are in the habit of doing, and pressed his suit with an earnestness and an ardor which left nothing to be desired. He pleased her; his absolute devotion flattered her. She fancied there was a sympathy of thought and taste between them, in which fancy she was mistaken. Add to this the violent opposition of her father and her sister Margaret to her marriage with a Catholic, and we need seek no further for the motives which led her to accept Monsieur Pontellier for her husband.

\indent The acme of bliss, which would have been a marriage with the tragedian, was not for her in this world. As the devoted wife of a man who worshiped her, she felt she would take her place with a certain dignity in the world of reality, closing the portals forever behind her upon the realm of romance and dreams.

\indent But it was not long before the tragedian had gone to join the cavalry officer and the engaged young man and a few others; and Edna found herself face to face with the realities. She grew fond of her husband, realizing with some unaccountable satisfaction that no trace of passion or excessive and fictitious warmth colored her affection, thereby threatening its dissolution.

\indent She was fond of her children in an uneven, impulsive way. She would sometimes gather them passionately to her heart; she would sometimes forget them. The year before they had spent part of the summer with their grandmother Pontellier in Iberville. Feeling secure regarding their happiness and welfare, she did not miss them except with an occasional intense longing. Their absence was a sort of relief, though she did not admit this, even to herself. It seemed to free her of a responsibility which she had blindly assumed and for which Fate had not fitted her.

\indent Edna did not reveal so much as all this to Madame Ratignolle that summer day when they sat with faces turned to the sea. But a good part of it escaped her. She had put her head down on Madame Ratignolle's shoulder. She was flushed and felt intoxicated with the sound of her own voice and the unaccustomed taste of candor. It muddled her like wine, or like a first breath of freedom.
\end{linenumbers*}

\newpage
\begin{enumerate}
\item Edna agreed to marry Monsieur Pontellier because

\begin{enumerate}[label=(\Alph*)]
\item They had similar tastes and ways of thinking.
\item Her family encouraged the marriage.
\item She felt obligated to make realistic choices.
\item His indifference made her crave his attention.
\item It was her destiny.
\end{enumerate}

\bigskip
\item Edna's feelings towards her husband can best be described as

\begin{enumerate}[label=(\Alph*)]
\item Passionate
\item Non-existent
\item Antagonistic
\item Moderately fond
\item Resentful 
\end{enumerate}

\bigskip
\item What ``muddled [Edna] like wine"?

\begin{enumerate}[label=(\Alph*)]
\item Sharing her ambivalent feelings about her life choices with Madame Ratignolle.
\item The exertion of bringing her children to the beach.
\item Her excessively passionate feelings for her husband.
\item Her uncertainty about the safety of her children.
\item The disapproval of her family towards her husband's religion.
\end{enumerate}
\end{enumerate}

\newpage
\textit{From Douglas Adams's novel A Hitchhiker's Guide to the Galaxy}

\bigskip
\begin{linenumbers*}
\modulolinenumbers[5]
\indent The Hitch Hiker's Guide to the Galaxy has a few things to say on the subject of towels. 

\indent A towel, it says, is about the most massively useful thing an interstellar hitch hiker can have. Partly it has great practical value - you can wrap it around you for warmth as you bound across the cold moons of Jaglan Beta; you can lie on it on the brilliant marble-sanded beaches of Santraginus V, inhaling the heady sea vapours; you can sleep under it beneath the stars which shine so redly on the desert world of Kakrafoon; use it to sail a mini raft down the slow heavy river Moth; wet it for use in hand-to-hand-combat; wrap it round your head to ward off noxious fumes or to avoid the gaze of the Ravenous Bugblatter Beast of Traal (a mind-boggingly stupid animal, it assumes that if you can't see it, it can't see you - daft as a bush, but very ravenous); you can wave your towel in emergencies as a distress signal, and of course dry yourself off with it if it still seems to be clean enough.

\indent More importantly, a towel has immense psychological value. For some reason, if a strag (strag: non-hitch hiker) discovers that a hitch hiker has his towel with him, he will automatically assume that he is also in possession of a toothbrush, face flannel, soap, tin of biscuits, flask, compass, map, ball of string, gnat spray, wet weather gear, space suit etc., etc. Furthermore, the strag will then happily lend the hitch hiker any of these or a dozen other items that the hitch hiker might accidentally have "lost". What the strag will think is that any man who can hitch the length and breadth of the galaxy, rough it, slum it, struggle against terrible odds, win through, and still knows where his towel is is clearly a man to be reckoned with.
\end{linenumbers*}

\bigskip
\begin{enumerate}
\item The tone of this passage can best be described as 

\begin{enumerate}[label=(\Alph*)]
\item Judgmental
\item Lyrical
\item Whimsical
\item Ribald
\item Contemplative
\end{enumerate}

\bigskip
\item The passage praises towels for having all of the following uses EXCEPT
\begin{enumerate}[label=(\Alph*)]
\item An advantage in a fight
\item Using as a parachute
\item Protection of the head
\item Extra clothing for warmth
\item Lying on the beach
\end{enumerate}

\newpage
\item The parenthetical statements within sentences serve the purpose of
\begin{enumerate}[label=(\Alph*)]
\item Contradicting the earlier claim of the sentence
\item Confusing the purpose of the sentence
\item Revealing the narrator's ambivalence toward the subject
\item Purposely complicating the prose.
\item Explaining the meaning of made-up words.
\end{enumerate}

\bigskip
\item What does the detail about how Monsier Pontellier ``pressed his suit" reveal?
\begin{enumerate}[label=(\Alph*)]
\item His disciplined and strict nature.
\item His aristocratic upbringing.
\item His sense of dreamy idealism.
\item His attempts to impress Edna's family.
\item His earnest worship of Edna.
\end{enumerate}
\end{enumerate}

\newpage
\textit{From NYTimes columnist David Brooks's The Art of Focus}

\bigskip
\begin{linenumbers*}
\modulolinenumbers[5]
\indent Like everyone else, I am losing the attention war. I toggle over to my emails when I should be working. I text when I should be paying attention to the people in front of me. I spend hours looking at mildly diverting stuff on YouTube. (“Look, there's a bunch of guys who can play ‘Billie Jean' on beer bottles!”) 

\indent And, like everyone else, I've nodded along with the prohibition sermons imploring me to limit my information diet. Stop multitasking! Turn off the devices at least once a week!

\indent And, like everyone else, these sermons have had no effect. Many of us lead lives of distraction, unable to focus on what we know we should focus on. According to a survey reported in an Op-Ed article on Sunday in The Times by Tony Schwartz and Christine Porath, 66 percent of workers aren't able to focus on one thing at a time. Seventy percent of employees don't have regular time for creative or strategic thinking while at work.

\indent Since the prohibition sermons don't work, I wonder if we might be able to copy some of the techniques used by the creatures who are phenomenally good at learning things: children.

\indent I recently stumbled across an interview in The Paris Review with Adam Phillips, who was a child psychologist for many years. First, Phillips says, in order to pursue their intellectual adventures, children need a secure social base:
``There's something deeply important about the early experience of being in the presence of somebody without being impinged upon by their demands, and without them needing you to make a demand on them. And that this creates a space internally into which one can be absorbed. In order to be absorbed one has to feel sufficiently safe, as though there is some shield, or somebody guarding you against dangers such that you can ‘forget yourself' and absorb yourself, in a book, say."

\indent Second, before they can throw themselves into their obsessions, children are propelled by desires so powerful that they can be frightening. ``One of the things that is interesting about children is how much appetite they have," Phillips observes. “How much appetite they have - but also how conflicted they can be about their appetites. Anybody who's got young children \ldots will remember that children are incredibly picky about their food. \ldots

\indent ``One of the things it means is there's something very frightening about one's appetite. So that one is trying to contain a voraciousness in a very specific, limited, narrowed way.\ldots An appetite is fearful because it connects you with the world in very unpredictable ways. \dots Everybody is dealing with how much of their own aliveness they can bear and how much they need to anesthetize themselves."

\indent Third, children are not burdened by excessive self-consciousness: ``As young children, we listen to adults talking before we understand what they're saying. And that's, after all, where we start - we start in a position of not getting it.” Children are used to living an emotional richness that can't be captured in words. They don't worry about trying to organize their lives into neat little narratives. Their experience of life is more direct because they spend less time on interfering thoughts about themselves.

\indent The lesson from childhood, then, is that if you want to win the war for attention, don't try to say ``no" to the trivial distractions you find on the information smorgasbord; try to say ``yes" to the subject that arouses a terrifying longing, and let the terrifying longing crowd out everything else.

The way to discover a terrifying longing is to liberate yourself from the self-censoring labels you began to tell yourself over the course of your mis-education. These formulas are stultifying, Phillips argues: ``You can only recover your appetite, and appetites, if you can allow yourself to be unknown to yourself. Because the point of knowing oneself is to contain one's anxieties about appetite."

\indent Thus: Focus on the external objects of fascination, not on who you think you are. Find people with overlapping obsessions. Don't structure your encounters with them the way people do today, through brainstorming sessions (those don't work) or through conferences with projection screens.

\indent Instead look at the way children learn in groups. They make discoveries alone, but bring their treasures to the group. Then the group crowds around and hashes it out. In conversation, conflict, confusion and uncertainty can be metabolized and digested through somebody else. If the group sets a specific problem for itself, and then sets a tight deadline to come up with answers, the free digression of conversation will provide occasions in which people are surprised by their own minds.

\indent The information universe tempts you with mildly pleasant but ultimately numbing diversions. The only way to stay fully alive is to dive down to your obsessions six fathoms deep. Down there it's possible to make progress toward fulfilling your terrifying longing, which is the experience that produces the joy.
\end{linenumbers*}

\begin{enumerate}
\item David Brooks works off the assumption that
\begin{enumerate}[label=(\Alph*)]
\item Children lack the technical skills to access the so-called information universe.
\item People should discourage their natural obsessions so they can focus better.
\item Discoveries are best made in groups of people who gathered to brainstorm.
\item Education allows people to discover and liberate their true natural inclinations and longings.
\item Distraction caused by technology is a universal experience that his readers all share.
\end{enumerate}

\bigskip
\item The word “stultifying” as used in the passage means
\begin{enumerate}[label=(\Alph*)]
\item Suppressive
\item Equivocal
\item Nurturing
\item Ambiguous
\item Disastrous
\end{enumerate}

\newpage
\item According to Brooks, the behavior of children should be
\begin{enumerate}[label=(\Alph*)]
\item Emulated
\item Corrected
\item Avoided
\item Celebrated
\item Adjusted
\end{enumerate}
\end{enumerate}