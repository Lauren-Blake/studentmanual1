\chapter{Reading Comprehension Part II}
\section{SAT Worksheet 1G: Warm-Up}
\textit{Read the following passage and answer the questions that follow}

\bigskip
\textit{From The Onion}
\begin{center}
\textbf{Woman Takes Short Half-Hour Break From Being Feminist To Enjoy TV Show}
\end{center}

\begin{linenumbers*}
\modulolinenumbers[5]
\indent Natalie Jenkins says she just wants to enjoy a little TV without thinking about how our culture repeatedly perpetuates gender stereotypes in a damaging way.

\bigskip
\indent PORTLAND, OR-Saying that she just wanted a little time to relax and ``not even think about" confining gender stereotypes, local health care industry consultant Natalie Jenkins reportedly took a 30-minute break from being a feminist last night to kick back and enjoy a television program.

\indent Jenkins, 29, told reporters that after a long and tiring day at her office, all she wanted to do was return home, sit down on her couch, turn on an episode of the TLC reality show \textit{Say Yes To The Dress}, and treat herself to a brief half hour in which she could look past all the various and near constant ways popular culture undermines the progress of women.

\indent
``Every once in a while, it's nice to watch a little television without worrying about how frequently the mainstream media perpetuates traditional gender roles," Jenkins said before putting her feet up on her coffee table and tuning in to the popular program that follows women as they shop for wedding gowns. ``No mentally cataloging all the times women are subtly mocked or shamed for not living up to an unrealistic body image, no examining how women are depicted as superficial and irrationally emotional, and no thinking about how these shows reinforce the belief that women should simply aspire to find a man and get married—none of that. Not tonight. I'm just watching an episode of Say Yes To The Dress and enjoying it for what it is."

\indent ``Between 9 and 9:30, I'm not even going to take notice of all the two-dimensional portrayals of women as fashion- and shopping-obsessed prima donnas," Jenkins added. ``That part of my brain will just be switched off."

\indent Jenkins confirmed that she watched contentedly for the entirety of the television program, telling reporters that she never once allowed herself to grow indignant as the adult, employed, and presumably self-respecting women on screen repeatedly demanded to be made into ``princesses."

\indent Additionally, Jenkins acknowledged that she witnessed dozens of moments in which the brides-to-be abandoned the notion that they should be valued for their personalities and intellects and instead seemed to derive their sole sense of worth from embellishing their appearance. However, she said she was able to consistently remind herself that this was ``Natalie time" and that the feminist movement ``could do without [her] for 30 minutes."

\indent ``Normally, I'd be pretty irritated at the thought of millions of people across the country mindlessly watching such a backward representation of what it means to be a woman in the 21st century, but tonight I'm just unwinding and not letting it get to me," Jenkins said. ``It's actually been kind of nice to push all the insinuations that marriage is the one true path for women to achieve happiness and fulfillment to the back of my mind and just lie back and have a good time." 

\indent ``In fact, there was a part where one of the brides threw a tantrum because the dress she wanted was above her budget and then whined to her father until he finally gave in and bought it for her, and I just let myself laugh out loud," added Jenkins, noting that, while she was fully aware that such depictions reinforced the notion of women as helpless figures who require a man to provide for them, she was “letting all that stuff slide” during this particular half hour. ``This show's actually pretty fun and entertaining if you ignore how damaging it could be to our perceptions of gender in society."

\indent Jenkins also reportedly viewed roughly 10 minutes of advertisements throughout the show, during which time she reminded herself to actively tune out the numerous instances wherein feminine sexuality was used to sell products; the number of times advertisements preyed on female insecurity; and the sheer volume of bare female skin shown on screen.

\indent ``Sure, I just watched several commercials that basically reduced women to explicitly sexualized objects whose sole purpose is to please men, but someone else can worry about that right now because I'm off the clock," said Jenkins, following a succession of ads for vodka, shampoo, and the Fiat 500. ``Honestly, I don't even care that that yogurt commercial showed thin, beautiful women easily balancing home and work lives while eating 60-calorie packs of yogurt. Tonight, in my mind, they're just selling Greek yogurt. That's all."

\indent While affirming that she had fully recommitted herself to the cause of gender equality as soon as the show's credits ended, Jenkins admitted she was already looking forward to the next time she could let herself disregard the many ways women are reduced to stale caricatures on national television.

\indent Honestly, it's pretty exhausting to call out every sexist stereotype or instance of misogyny in popular culture, so sometimes I have to just throw my hands up and grant myself a little time off,” Jenkins said. ``And given the state of modern media, momentarily suspending my feminist ideals is the only way to get through a night of TV without becoming totally livid or discouraged."

\indent As of press time, Jenkins' sense of relaxation and contentment had been entirely undone by the first 30 seconds of 2 Broke Girls.
\end{linenumbers*}

\begin{enumerate}
\item \textbf{How would you describe the style and tone of this article?}
\vfill\item \textbf{Does Natalie seemed informed on issues of gender in society?}
\vfill\item \textbf{Do you think she succeeds in ``taking a break from being a feminist"? Why or why not?}
\end {enumerate}

\vfill
\newpage
\section{Paired Passages}
The SAT critical reading section also includes sections with 2 passages. The paired long passages can be the most intimidating part of the critical reading because they are relatively long and therefore, look intimidating. However, the strategies and practice in this chapter will help you to succeed on the sections with paired passages.

\vfill
\section[Intro to Paired Passages]{SAT Worksheet 2G: Introduction to Paired Passages}
\textit{Directions: Read the following italicized passages that would appear before a passage. Then, answer the questions that follow.}
The first passage about the Washington Memorial is authored by a historian, whereas the second is authored by a civil engineer.
\begin{enumerate}
\item How could the author's professions lead to differences in the passages?

\hrulefill
\item What topics or point of view is the historian likely to present?

\hrulefill
\item What topics or point of view is the historian likely to present?

\hrulefill
\end{enumerate}


The first passage is from an essay entitled ``Why should boys have all the fun?: Science and engineering toys for girls". The second passage is from an essay entitled ``Call for mentoring girls interested in the STEM (science, technology, engineering, and mathematics) fields."

\begin{enumerate}
\item Based on the titles, what do you think that the two passages will have in common?

\hrulefill
\item Based on the titles, what do you think that the two passages may differ on? Why?

\hrulefill
\end{enumerate}

\vfill
\newpage
\section[Strategies]{Strategies for Paired Passages}

\begin{enumerate}[leftmargin=0cm,labelwidth=\itemindent,labelsep=0cm,align=left,label={\bfseries Strategy \#\arabic*:\ }]
\item Read the italicized headnote at the beginning of the two passages.  This information will often have clues about the purpose, origin, writer, or setting of the excerpt.  It will help you make predictions, differentiate between the two passages, and understand the perspective of each.

\bigskip
For example, there are two passages discussing the legalization of gay marriage.  The italicized headnote tells you the first one is written by a social liberal and the second by a social conservative.  This introductory information may give you some insight into their likely stance on the topic even before reading the excerpts.  

\item The questions go in order according to how the passages are presented. 

\bigskip
Read passage 1 and answer the questions about passage 1 (the beginning questions). Read passage 2 and answer the question about the passage 2 (the middle questions). Then, think about the relationship between the passages—there will more about this in later strategies—and answer the questions about the passages together (the questions at the end).

\item The questions that ask you to analyze and evaluate the two passages and how they relate to each other will usually come closer to the end of the questions about the two passages.  When you read, ask if they agree or disagree with each other.  
\end{enumerate}

\bigskip
Another possible relationship is that the first passage introduces a topic generally while the second elaborates on it in more detail.  As you're reading, pay attention to the overall tone, purpose, and stance of the passages.

\bigskip
Some relationships between passage 1 and 2 may be characterized by the following:

\begin{itemize}
\item Passage 2 provides evidence that proves the argument made in Passage 1.
\item Passage 2 elaborates on claims made in Passage 1.
\item Passage 2 exposes the flaws in the argument made in Passage 1.
\item Passage 2 provides an exception to the rule established in Passage 1.
\item Passage 2 contradicts the opinion presented in Passage 1.
\end{itemize}

\bigskip
\textbf{For example, }

Passage 1: Posits that human-caused global warming will have catastrophic consequences on our environment within the next century.

Passage 2: Suggests that the earth has always gone through extreme climate changes, and global warming is similar to those that occur naturally.

\newpage
\textbf{Compare and contrast questions usually ask the following:}

\bigskip
\indent \textbf{``How would the author of passage 2 likely respond to the author of passage 1 in his claim that human activity is destroying the environment on a global scale"}- This general type of question is based on inference.  You have to interpret the authors' opinions and infer author 2's response based on the opinion (s)he presents in the excerpt.
    
\bigskip
\indent \textbf{``Which of the following is a view expressed by both passages?"}   Your answer must be closely based on the text.  Look at the answer choices and find a generalized idea that both authors would agree with.

\bigskip
\indent \textbf{``Which of the following is a difference between passage one and two?"} This is asking you to contrast the two passages.  It does not necessarily require you to make an inference, but does require that you understand the perspectives of both authors.

\newpage
\section[Paired Passages]{SAT Worksheet 3G: Paired Passages Practice Questions}
\textit{Directions: Read Passage 1 and answer the question \#1 at the end (about passage 1). Then, read passage 2 and answer the question \#2 (about passage 2). Finally, answer the questions that refer to both questions. }

\bigskip
\textbf{Passage 1}

\textit{Susan Patton's Letter to the Editor: Advice for the young women of Princeton: the daughters I never had}

\bigskip
\begin{linenumbers*}
\modulolinenumbers[5]
\indent Forget about having it all, or not having it all, leaning in or leaning out - here's what you really need to know that nobody is telling you.

\indent For years (decades, really) we have been bombarded with advice on professional advancement, breaking through that glass ceiling and achieving work-life balance. We can figure that out - we are Princeton women. If anyone can overcome professional obstacles, it will be our brilliant, resourceful, very well-educated selves.

\indent A few weeks ago, I attended the Women and Leadership conference on campus that featured a conversation between President Shirley Tilghman and Wilson School professor Anne-Marie Slaughter, and I participated in the breakout session afterward that allowed current undergraduate women to speak informally with older and presumably wiser alumnae. I attended the event with my best friend since our freshman year in 1973. You girls glazed over at preliminary comments about our professional accomplishments and the importance of networking.

\indent Then the conversation shifted in tone and interest level when one of you asked how have Kendall and I sustained a friendship for 40 years. You asked if we were ever jealous of each other. You asked about the value of our friendship, about our husbands and children. Clearly, you don't want any more career advice. At your core, you know that there are other things that you need that nobody is addressing. A lifelong friend is one of them. Finding the right man to marry is another.

\indent When I was an undergraduate in the mid-seventies, the 200 pioneer women in my class would talk about navigating the virile plains of Princeton as a precursor to professional success. Never being one to shy away from expressing an unpopular opinion, I said that I wanted to get married and have children. It was seen as heresy.

\indent For most of you, the cornerstone of your future and happiness will be inextricably linked to the man you marry, and you will never again have this concentration of men who are worthy of you.

\indent Here's what nobody is telling you: Find a husband on campus before you graduate. Yes, I went there.

\indent I am the mother of two sons who are both Princetonians. My older son had the good judgment and great fortune to marry a classmate of his, but he could have married anyone. My younger son is a junior and the universe of women he can marry is limitless. Men regularly marry women who are younger, less intelligent, less educated. It's amazing how forgiving men can be about a woman's lack of erudition, if she is exceptionally pretty.

\indent Smart women can't (shouldn't) marry men who aren't at least their intellectual equal. As Princeton women, we have almost priced ourselves out of the market. Simply put, there is a very limited population of men who are as smart or smarter than we are. And I say again — you will never again be surrounded by this concentration of men who are worthy of you.

\indent Of course, once you graduate, you will meet men who are your intellectual equal — just not that many of them. And, you could choose to marry a man who has other things to recommend him besides a soaring intellect. But ultimately, it will frustrate you to be with a man who just isn't as smart as you.

\indent Here is another truth that you know, but nobody is talking about. As freshman women, you have four classes of men to choose from. Every year, you lose the men in the senior class, and you become older than the class of incoming freshman men. So, by the time you are a senior, you basically have only the men in your own class to choose from, and frankly, they now have four classes of women to choose from. Maybe you should have been a little nicer to these guys when you were freshmen?

\bigskip
If I had daughters, this is what I would be telling them.
\end{linenumbers*}

-Susan A. Patton '77

\bigskip
\textbf{Passage 2}

\textit{Claire Fallon's The 10 Worst Pieces of Advice from Susan Patton's Marry Smart}

\bigskip
\begin{linenumbers}
\modulolinenumbers[5]
Susan Patton, also known as “The Princeton Mom,” first caught the public eye in March 2013, when she published a letter to the editor in The Daily Princetonian. The letter advised the young female students at Patton's alma mater to seek husbands while at Princeton rather than dating the lower-quality men they'd meet in their post-college lives, and to dedicate more of their time and energy to finding a good husband rather than focusing on their careers. Less than one year after that initial media circus, and several weeks after one wisely timed repeat performance in a Wall Street Journal op-ed last month, Patton has returned with a full-length book version of her original advice, Marry Smart: Advice for Finding the One. The 11-month turnaround suggests a rush to capitalize on her brush with the limelight, and indeed the quality of the book does seem as slapdash as could be expected.

\indent Of course, we could have hoped that Patton's opus, when it emerged, would be less repetitive, more polished, and less replete with awkward logical fallacies. My boyfriend, a state school grad, writes text messages more finely crafted and coherent than her latest admonition to seek out husbands with Ivy League degrees. But it's not the clunky prose or the endless redundancies that doomed the book from the beginning, and even a fine-tuned version would have only succeeded in putting a prettier face on her flawed advice. The real problem was trying to turn one page of clichéd sexist tropes and ugly elitism disguised as advice into 200+ pages (238, if we're counting) of constructive tips for young women today.

\indent I'm right in the target audience for Susan Patton's advice. I'm 25, an alumna of her cherished Princeton, and still not married. During my single years in New York City, I spent considerably more time working and considering my career options than dating or angling to meet new men. Patton clearly tries to preemptively extinguish criticism about the sexist roots of her advice by repeatedly assuring us that her advice is only for women who want to have children and ``something resembling a traditional marriage." Well, I want both -- surprise, I'll admit that despite having been brainwashed by feminists! -- so \ldots did I find Marry Smart to be just the no-nonsense straight talk that I needed to achieve my true dreams of Leave-It-To-Beaver-style domestic bliss?

\indent Well, if you define ``straight talk" as ``hideous sexist stereotypes that were outdated 20 years ago," then sure. But I can't say any of the advice actually seems useful or relevant to me, a 20-something in 2014. The only wise tidbits are so trite they hardly needed to be reiterated yet again -- e.g., get involved with activities you care about and date men with whom you share core values. And a lot of her more outr\'e advice seemed downright laughable.
\end{linenumbers}

\begin{enumerate}
\item How would you describe the tone of Passage 1? 
\vfill\item How would you describe the tone of Passage 2?  
\vfill\item What general ideas do the authors of Passage 1 and 2 agree upon, if any?
\vfill\item What is the biggest difference in the viewpoints of the author of between Passage 1 and of Passage 2?
\vfill\item What is the author of Passage 1's advice to the `daughters of Princeton'?  How does the author of Passage 2 feel about this advice?
\vfill\item How would the author of Passage 1 feel about the author of Passage 2 dating ``a state school grad"? 
\end{enumerate}

\vfill
\newpage
\section[Analyzing Incorrect]{Analysis of Incorrect Answers}
\textit{Analyze the passage-based reading questions that you got wrong in class or for homework and fill out the chart below.}

\bigskip
\begin{tabularx}{\textwidth}{|X|p{2in}|p{2in}|X|}\hline
Section and Question \# & Reason why the answer I originally selected is wrong & Evidence for the correct answer (line \# and phrase) & New Answer Selected\\\hline
& & &\\[8ex]\hline
& & &\\[8ex]\hline
& & &\\[8ex]\hline
& & &\\[8ex]\hline
& & &\\[8ex]\hline
& & &\\[8ex]\hline
& & &\\[8ex]\hline
& & &\\[8ex]\hline
& & &\\[8ex]\hline
& & &\\[8ex]\hline
\end{tabularx}

\section[Paired Passages]{SAT Worksheet 4G: Paired Passages Practice Questions}
\textit{Directions: Read the following passages and answer the questions that follow.}

\bigskip
\textbf{The first passage is entitled ``Ain't I a Woman?" by Sojourner Truth, an African American activist. Passage 2 is taken from an address by Susan B. Anthony, a suffragist.}

\bigskip
\textbf{Passage 1}

\bigskip
\begin{linenumbers*}
\modulolinenumbers[5]
\indent That man over there says that women need to be helped into carriages, and lifted over ditches, and to have the best place everywhere. Nobody ever helps me into carriages, or over mud-puddles, or gives me any best place! And ain't I a woman? Look at me! Look at my arm! I have ploughed and planted, and gathered into barns, and no man could head me! And ain't I a woman? I could work as much and eat as much as a man - when I could get it - and bear the lash as well! And ain't I a woman? I have borne thirteen children, and seen most all sold off to slavery, and when I cried out with my mother's grief, none but Jesus heard me! And ain't I a woman?
\end{linenumbers*}

\bigskip
\textbf{Passage 2}

\bigskip
\begin{linenumbers}
\modulolinenumbers[5]
\indent The only question left to be settled, now, is: Are women persons? And I hardly believe any of our opponents will have the hardihood to say they are not. Being persons, then, women are citizens, and no state has a right to make any new law, or to enforce any old law, that shall abridge their privileges or immunities. Hence, every discrimination against women in the constitutions and laws of the several states, is to-day null and void, precisely as is every one against negroes.

\indent Is the right to vote one of the privileges or immunities of citizens? I think the disfranchised ex-rebels, and the ex-state prisoners will agree with me, that it is not only one of the them, but the one without which all the others are nothing. Seek the first kingdom of the ballot, and all things else shall be given thee, is the political injunction.
\end{linenumbers}

\bigskip
\begin{enumerate}
\item It can be inferred that both authors
\begin{enumerate}[label=(\Alph*)]
\item Originate from a place of socio-economic privilege
\item Share a satisfaction with the status quo
\item Are primarily concerned with legal reform
\item Are activists concerned with gender inequality
\item Are radicals that represent a threat to their societies
\end{enumerate}

\newpage
\item The relationship between the two passages can best be described as
\begin{enumerate}[label=(\Alph*)]
\item Similar in ideological stance and complementary in style
\item Similar in ideological stance but different in style
\item Different in ideological stance but similar in style
\item Opposing in ideological stance and different in style
\item Contradictory in both ideological stance and style
\end{enumerate}

\bigskip
\item The author of passage 1 differs from the author of passage 2 in that she
\begin{enumerate}[label=(\Alph*)]
\item Uses colloquial language and personal experience
\item Favors racial equality to gender equality
\item Expresses herself in an elevated and inaccessible manner
\item Alienates her audience with targeted criticism
\item Questions the authenticity of other female activists
\end{enumerate}

\bigskip
\item The author of passage 2 expresses concern with the equal rights of all groups of people except
\begin{enumerate}[label=(\Alph*)]
\item Women
\item African Americans
\item Soldiers
\item Ex-rebels
\item Ex-state prisoners
\end{enumerate}

\bigskip
\item When the author of passage 1 says ``nobody...gives me any best place," she means
\begin{enumerate}[label=(\Alph*)]
\item She has experienced but been unimpressed with the luxuries of the upper class.
\item She has not had the opportunity to experience ways of life in different cities and regions.
\item She has not been afforded the privileges enjoyed by women of higher socio-economic status.
\item She has rejected the help of men as unwelcome and condescending.
\item She feels she is less of a woman because she has not experienced the usual niceties extended to women.
\end{enumerate}

\end{enumerate}

\bigskip
The passages below discuss the personality and achievements of Christopher Columbus.

\bigskip
\textbf{Passage 1}

\bigskip
\begin{linenumbers*}
\modulolinenumbers[5]
\indent Christopher Columbus was born in Genoa between August and October 1451. His father was a weaver and small-time merchant. As a teenager, Christopher went to sea, travelled extensively and eventually made Portugal his base. It was here that he initially attempted to gain royal patronage for a westward voyage to the Orient - his `enterprise of the Indies'.

\indent When this failed, and appeals to the French and English courts were also rejected, Columbus found himself in Spain, still struggling to win backing for his project. Finally, King Ferdinand and Queen Isabella agreed to sponsor the expedition, and on 3 August 1492, Columbus and his fleet of three ships, the Santa Maria, the Pinta and the Niña, set sail across the Atlantic.

\indent Ten weeks later, land was sighted. On 12 October, Columbus and a group of his men set foot on an island in what later became known as the Bahamas. Believing that they had reached the Indies, the newcomers dubbed the natives `Indians'. Initial encounters were friendly, but indigenous populations all over the New World were soon to be devastated by their contact with Europeans. Columbus landed on a number of other islands in the Caribbean, including Cuba and Hispaniola, and returned to Spain in triumph. He was made 'admiral of the Seven Seas' and viceroy of the Indies, and within a few months, set off on a second and larger voyage. More territory was covered, but the Asian lands that Columbus was aiming for remained elusive. Indeed, others began to dispute whether this was in fact the Orient or a completely `new' world.

\indent Columbus made two further voyages to the newfound territories, but suffered defeat and humiliation along the way. A great navigator, Columbus was less successful as an administrator and was accused of mismanagement. He died on 20 May 1506 a wealthy but disappointed man.
\end{linenumbers*}

\textit{From BBC History}

\bigskip
\textbf{Passage 2}

\bigskip
\begin{linenumbers}
\modulolinenumbers[5]
\indent In 5,000 years of recorded history, scarcely another figure has ignited as much controversy. Each second Monday in October, the familiar arguments flare up. Christopher Columbus, rediscoverer of America, was a visionary explorer. He was a harbinger of genocide. He was a Christianizing messiah. He was a pitiless slave master. He was a lionhearted seaman, a rapacious plunderer, a masterly navigator, a Janus-faced schemer, a liberator of oppressed tribes, a delusional megalomaniac. In ``Columbus," Laurence Bergreen, the author of several biographies, allows scope for all these judgments. But Christopher Columbus was in the first place a terribly interesting man -- brilliant, audacious, volatile, paranoid, narcissistic, ruthless and (in the end) deeply unhappy.
\end{linenumbers}

\newpage
\begin{enumerate}
\item The tone of passage 1 could best be described as
\begin{enumerate}[label=(\Alph*)]
\item Hypothetical and hypocritical
\item Fact-based and biased
\item Biographical and straightforward
\item Subjective and disenchanted
\item Attentive and venerational
\end{enumerate}

\bigskip
\item The relationship between the two passages could best be described as
\begin{enumerate}[label=(\Alph*)]
\item Passage 1 gives a summary while Passage 2 offers analysis
\item Passage 1 is theoretical while Passage 2 gives concrete examples
\item Passage 1 introduces while Passage 2 elaborates
\item Passage 1 is nuanced while Passage 2 is more accessible
\item Passage 1 editorializes while Passage 2 promotes
\end {enumerate}

\bigskip
\item What best describes the attitude the author takes toward Christopher Columbus in passage 2?
\begin{enumerate}[label=(\Alph*)]
\item Multitudinous and reproachful
\item Extolling and respectful
\item Discombobulated and disgruntled
\item Holistic and complex
\item Radical and pragmatic
\end{enumerate}


\bigskip
\item In line 23, ``harbinger" most closely means
\begin{enumerate}[label=(\Alph*)]
\item Person who conquers other civilizations
\item Person who explores new lands
\item Person who navigates across the globe
\item Person who signals the approach of something
\item Person who opposes an action
\end{enumerate}

\newpage
\item The author of Passage 2 would most likely consider Passage 1 to be
\begin{enumerate}[label=(\Alph*)]
\item Illusory and immutable
\item Sagacious and comprehensive
\item Exceptionable and illuminating
\item Lacking and inadequate
\item Accurate and introductory
\end{enumerate}
\end{enumerate}