\chapter[Multiple Choice Part I]{SAT Writing Multiple Choice Part I}

\section{SAT Worksheet 1A: Warm-Up}
\textit{Today, we will begin studying and unlocking the secrets of doing well on the SAT writing multiple choice sections. To help you and your instructor better assess your background in this area, please answer the following questions.}

\bigskip
Reflect on your most recent written assignments (ranging from one paragraph to multiple pages). What grammatical concepts do you think that you know and execute well?




\vfill

What grammar concepts do you have the most trouble with, that is to say, what types of errors do you make the most frequently? 

\vfill
\pagebreak
\section[Types of Questions]{Types of Questions on the Writing Multiple Choice Section}

The multiple choice section on the writing Section Multiple Choice has three types of questions: error identification, improving sentences, and improving paragraphs.

\bigskip
\textbf{In sentence error questions,} you will read a sentence and circle the part of the sentence that contains the error. There will also be an option for no error. 

\bigskip
For example:

\begin{inparaenum}[A]
\begin{spacing}{1.25}
\begin{tabularx}{\textwidth}{*8{@{}>{\ }c}}
Every day & Maria and & me & go & to the store & \ & to buy milk, & bread, and cheese.  \\\cline{1-1}\cline{3-3}\cline{5-5}\cline{7-7}
\item &  &\item & & \item & &\item &  \\
\end{tabularx}

\begin{tabularx}{\textwidth}{*1{@{}>{\ }c}}
No error
\\\cline{1-1}
\item\\
\end{tabularx}

\end{spacing}

\end{inparaenum}

What would you select here as the error?

\bigskip

\bigskip

\bigskip

\bigskip

\bigskip

\textbf{In sentence improvement questions,} you will be asked to select the best version of a sentence.

\bigskip
For example:\\
Every day, Maria and me go to the store to buy milk, cheese, and bread.
\begin{enumerate}[label=(\Alph*)] \itemsep-0.4em
\item{Every day, Maria and me go to the store to buy milk, cheese, and bread.}
\item{Every day, Maria and me are going to the store to buy milk, cheese, and bread.}
\item{Every day, Maria and I are going to the store to buy milk, cheese, and bread.}
\item{Every day, Maria and I go to the store to be buying milk, cheese, and bread.}
\item{Every day, Maria and I go to the store to buy milk, cheese, and bread.}
\end{enumerate}


\bigskip

\bigskip
What is the best version of the original sentence? Remember, the best answer must preserve the meaning of the original sentence, be free of grammatical errors, and be concise.


\bigskip

\bigskip

\bigskip

\bigskip

\bigskip

\textbf{In the paragraph improvement questions,} the SAT will present you with a paragraph and then ask you to improve the questions at a sentence level (similar to the paragraph improvement questions) or the paragraph level. They can ask about editing, moving, deleting, or adding a sentence to the paragraph in order to increase the clarity.

\bigskip
For example: \\
(Sentence 1) Most parents and students stress about getting accepted to the colleges of their choice, but it can be very difficult to figure out what colleges are really looking for. (2) There are other factors that are also worthy of student's and parent's attention. (3) For example, did you know that community service, extra-curricular activities, a strong personal essay, or a display of interest in the particular school can all help you get accepted? (4) Unlike in other countries which rely solely on grades and an admission test, admissions to American universities are holistic. 

\bigskip

\bigskip
Which of the following improvements should be made to the paragraph above?

\begin{enumerate}[label=(\Alph*)]\itemsep-0.4em    %I changed this so that the question and answer choice were so spread out.I do want to keep the page break after this question.
\item{Keep as is}
\item{Move sentence 1 to before sentence 4}
\item{Add “In addition to SAT scores and grades” to the beginning of sentence 2}
\item{Change “are” to “were” in sentence 2}
\item{Delete sentence 3 }
\end{enumerate}

\pagebreak

Before we discuss the strategy of for each of the types of questions, we are going to focus on what types of writing and grammar topics are tested in each section and also the rules for each of these grammar points: 

\bigskip
\begin{center}
\textbf{\underline{Error Identification}}\\

\bigskip
Subject-verb agreement\\
Pronoun reference\\
Parallelism\\
Adverbs vs. Adjectives\\
Tenses\\
Singular-Plural Noun Inconsistency\\
Comparatives vs. Superlatives\\
Sentence Fragments\\
Shift in Point of View\\
You and Me Errors\\
Idioms\\
Redundancy\\
Word Choice\\
Hypothetical Statements\\

\bigskip
\textit{Error Identification questions rarely test rules regarding run-ons and modifiers.}

\bigskip

\textbf{\underline{Sentence Improvements}}\\

\bigskip
Subject-Verb Agreement\\
Pronoun Reference\\
Run-ons\\
Modifiers\\
Parallelism\\
Shift in Point of View\\
You and Me Errors\\
Idioms\\
Redundancy\\
Hypothetical Statements\\
Sentence fragments\\

\bigskip
\textit{Improving sentences questions rarely test rules regarding adverbs vs. adjectives, singular-plural noun inconsistency, comparatives vs. superlatives, or word choice.} 

\bigskip
\textbf{\underline{Paragraph Improvements}}\\
\bigskip
Subject-Verb Agreement\\
Pronoun Reference\\
Run-ons\\
Modifiers\\
Parallelism\\
Sentence Fragments\\
Tenses\\
Idioms\\
Redundancy\\
Shift in Point of View (1st person, 2nd person, 3rd person, etc.)\\

\bigskip
\textit{Improving paragraphs questions rarely test rules regarding adverbs vs. adjectives, singular-plural noun inconsistency, comparatives vs. superlatives, you and me errors, word choice, or hypothetical statements.}

\end{center}

\bigskip

\bigskip

\section[Grammar Topics]{SAT Worksheet 2A: Grammar Topics Frequently Tested on the SAT Writing Section}
Directions: Read about each of the grammar rules below. Fill in the blanks with the correct response(s) according to the rules described.


\bigskip
\textbf{\large Subject-Verb Agreement}


\subsection{Subject - Non-essential clause - Verb}
These sentences often insert a non-essential clause, set off by commas, between the subject and verb to distract you.

\bigskip
Ex: Apple computers, though popular in the United States, is/are less common in other countries.

\bigskip
‘Though popular in the United States' is a parenthetical clause.  It can be removed from the sentence without affecting its overall meaning.  In this case, it separates the subject “Apple computers” from the verb “are”.\\
In these kind of sentences, just cross out the parenthetical clause and check that the verb agrees with the subject.

\subsection{Subject - Prepositional Phrase - Verb}
A prepositional phrase begins with a preposition, such as with, from, to, of, in, on, and over.  When prepositional phrases sit between subjects and verbs, they can distract from subject-verb agreement.

\bigskip
\textbf{Circle the correct verb:} 
Changes in the temperature of the Earth seem/seems small, but even small changes have a huge environmental impact.


\subsection{Prepositional Phrase - Verb - Subject}
In this case, the subject comes near the end of the sentence rather than the beginning. Make sure to identify if the subject is singular or plural.


\bigskip
\textbf{Circle the correct verb:}
\begin{enumerate}\itemsep-0.4em
\item{Along the Charles River is/are many runners and cyclists.}
\item{Along the Charles River is/are a runner and a cyclist.}
\item{Along the Charles River is/are a runner.}
\end{enumerate}

\subsection{There is/There are, There has/There have}
There is/has = Singular noun
There are/have = Plural noun

\bigskip
\textbf{Circle the correct verb:} 
There has/have been many questions and few answers about the missing plane.

\subsection{Neither/Nor + Verb}
The verb always agrees with the noun after ``nor.''

\bigskip
\textbf{Answer the following questions about the sentence:}
Neither Mark nor his brother plays/play an instrument.\\
What is the noun after ``nor''? \hrulefill \\
Is this noun singular or plural? \hrulefill

\bigskip
Important:\\
Collective Nouns (e.g. company, school, city, country, committee, jury, agency etc.) are singular.

\begin{enumerate}
\item{Each, Every, One = Singular}
\item{A singular noun, followed by ``in addition to,'' ``as well as," or ``including''} remains singular.
\item{A number (of) = Singular}
\item{(N)either one OR whether (n)either clearly refers to two singular nouns = Singular}
\item{Gerunds when used as subjects (e.g. Taking standardized tests often takes several hours) = Singular.}
\item{What and whether as subjects (e.g.``Whether dogs or cats are better is a subject of debate for some people."); both are singular.}
\end{enumerate}

\bigskip
\textbf{Practice: Circle the correct verb for the sentences below.}
\begin{enumerate} 
\item{The restaurants near the club which host/hosts celebrity parties stay/stays open until 3 am.}
\item{The polka dot pattern of our skirts reflect/reflects the trend this season.}
\item{Each group of four students has/have to make a PowerPoint for their presentation.}
\item{My neighbor with all the cats walk/walks down the street every morning.}
\item{Everybody take/takes a writing class in the first semester.}
\item{The people who watch that show is/are few.}
\item{The boat captain, as well as his crew, is/are highly trained.}
\item{The beginning of the book, including chapters one through five, is/are boring.}
\item{Five dollars is/are a lot of money.  U.S. dollars is/are worth less than euros.}
\item{Neither my sister nor I enjoy/enjoys seafood.}
\end{enumerate}

\bigskip

\bigskip
\section{Pronoun Reference}
\subsection{Pronouns Must Have Clear Antecedents}
Singular nouns go along with singular pronouns, such as he, she, his, her, it, its.
Plural nouns go along with plural pronouns, like them and their.

\bigskip
\textbf{For example:}
A person who wants to play professional sports must spend all of his/her time practicing.

\bigskip
People who want to play professional sports must spend all of their time practicing.

\bigskip
A pronoun must have a clear antecedent, the noun, pronoun, or gerund (e.g. swimming, reading, eating) to which it is referring.

\bigskip
Wrong: Because of the severe drought, they don't have enough to drink.
\textit{They has no clear antecedent.}

\bigskip
Fix it: \hrulefill

\bigskip
Wrong: In the essay, it analyzed the themes of resistance to totalitarian rule.
\textit{It has no clear antecedent.}

\bigskip
Fix it: \hrulefill

\bigskip
Wrong: Katy and Lucy are going to take her car.
\textit{Her has no clear antecedent.}
Fix it: \hrulefill

\subsection{Do So vs. Do it}

Do it = Wrong \\
Do so = Right 

\bigskip
For example: People who travel do it because they love exploring.\\
What does `it' refer to in this sentence? Traveling. But since the gerund `traveling' doesn't actually appear in the sentence, `it' has no real antecedent.

\bigskip
\textit{Important:
For both Subject-Verb Agreement and Pronoun Agreement, be on the lookout for collective nouns such as family, group, committee, jury, city, agency, team, etc. These nouns are always singular, and it is not uncommon for the SAT to pair them with plural verbs and pronouns. Whenever one of these words appears, you should immediately be suspicious.\\
In sentence error questions, ``it'' is often wrong.  If ``it'' is underlined, check its antecedent immediately!}

\bigskip
\textbf{Practice: Circle the pronoun error. Then, give a possible correction.}
\begin{enumerate}
\item{Whenever Katy and Sara go out to dinner, she pays for the check.}
\item{Even if a student has perfect grades, they have no guarantee of getting into Harvard.}
\item{At the zoo, they saw lions, zebras, and giraffes.}
\item{He always drives on empty, and this really bothers me!}
\item{Although it is a lot bigger, elephants can still be hunted and eaten by lions.}
\end{enumerate}


\section{Run-ons}
A comma cannot connect two independent clauses.  This error is called a ``comma splice.'' Check if the two sentences should be separated by a period, a semi-colon, or a transition word like ``so''.

\bigskip
\textbf{Practice: Correct the following sentences. }
\begin{enumerate}
\item{Thank you for your consideration, I look forward to hearing from you.}
\item{In Europe, the cafes are very relaxed, people can sit for as long as they like.}
\item{Karlee was interested in international relations, she applied to graduate programs all over the world.}
\end{enumerate}

\section{Dangling Modifiers} 
Modifiers are words or phrases which modify other elements in the sentence, usually nouns.  They usually can be removed from the sentence without affecting the structure of the sentence overall.

\bigskip
Incorrect: Grabbing a towel, the showers were my first stop. \\
Correct: Grabbing a towel, I headed for the showers. 

\bigskip
In this sentence, “grabbing a towel” is the modifier.  It modifies “I”.  When it is placed next to “the showers,” it sounds like the showers grabbed a towel.  That is impossible.

\bigskip
Incorrect: Having quit her job to travel, Ecuador was Sophie's first stop.  Ecuador didn't quit its job to travel!  Sophie did.

\bigskip
Fix it: \hrulefill


\bigskip
\textbf{Practice: Correct the following sentences. At least two should be corrected by placing the modifier at the beginning of the sentence ending with a comma and the subject immediate after the comma.}

\begin{enumerate}
\item{Krystal saw three orange cars jogging down the street.}

\item{The teacher impressed his students playing ultimate Frisbee.}

\item{Decorated with lights, we took pictures in front of the town Christmas tree.} 


\item{After scoring the game-winning goal in the last minute, the crowd cheered for the hockey player.}
\end{enumerate}

\section{Parallel Structure}
Use the same pattern of words at the word, phrase, or clause level. Read about parallel structure and fill in the blanks where provided.

\subsection{At the word level:}
Incorrect: I like swimming, hiking, and to raft.\\
Correct: I like swimming, hiking, and rafting.\\
Correct:  I like to swim, hike, and raft.\\
Also correct:  I like to swim, to hike, and to raft.

\subsection{At the phrase level:}
Incorrect:  The homework should be done quickly, accurately, and in a thorough manner.\\
Correct: The homework should be done quickly, accurately, and thoroughly.

\bigskip
Incorrect:  The teacher praised him because of his hard work, motivation, and he was conscientious.\\
Correct: The teacher praised him because of his hard work, motivation, and
\longline   %The previous command made a line that was too dark. I think that this command is correct but please check.

\subsection{At the clause level:}
Incorrect: To prepare for this test, you should study, get a good night's sleep, and eating breakfast is important. \\
Correct: To prepare for this test, you should study, get a good night's sleep, and eat breakfast.\\
Also correct:  To prepare for this test, you should study, you should get a good night's sleep, and you should eat breakfast.

\bigskip
Incorrect:  The teacher expected that his students would pay attention, do their homework, and that questions would be asked.\\
Correct:  The teacher expected that his students would pay attention, do their homework, and \hrulefill \\
Also correct:  The teacher expected that his students would pay attention, would do their homework, and \hrulefill.

\subsection{In a list after a colon}
Incorrect:  Remember to pack the following: toothbrush, change of clothes, and you will also need hiking boots.

\bigskip
Correct:  Remember to pack the following: toothbrush, change of clothes, and hiking boots.

\subsection{With common pairs}
\begin{itemize}
\item{Prefer\ldots to\ldots}
\item{(Decide) between\ldots and\ldots}
\item{Not only to\ldots but also to\ldots}
\item{Neither\ldots nor\ldots}
\item{Either\ldots or\ldots}
\item{As\ldots as\ldots}
\end{itemize}

\subsection{With comparisons}

Incorrect: Americans drive bigger cars than European countries.\\
Correct:  Americans drive bigger cars than people in European countries.

\bigskip
Incorrect:  I like jogging more than hikes outside.\\
Correct: I like jogging more than \hrulefill.

\bigskip
\textbf{Practice: Circle the word(s) that do not follow parallel structure. Then, correct the sentence so that it follows parallel structure.}

\begin{enumerate}
\item{I respect your intelligence and that you are eloquent.}
\item{To improve reading comprehension, remember to take notes, to draw inferences, and summarizing.}
\item{Doing well on the SATs requires to study for months.}
\item{My little sister prefers macaroni and cheese over broccoli.}
\item{It's hard to decide between going on vacation or saving money.}
\item{The movie is not so funny as everyone insists it is.}
\item{To write poetry is appreciating the details and beauty in your surroundings.}
\end{enumerate}

\section{Adverbs vs. Adjectives}
Adjectives describe nouns.\\
For example: Tall man, beautiful flower, miraculous recovery.  

\bigskip
Adverbs describe verbs, adjectives, or other adverbs.  Adverbs often end in –ly.\\
Smile happily, run quickly, work confidently, do well, act fast, amazingly fast runner.

\bigskip
\textbf{Practice: In the following sentences, correct the adjective or adverb errors.}
\begin{enumerate}
\item{Meghan, who did not feel confident about her performance, received an astonishing high score on her English final.}
\item{Rapid advancing technology has completely transformed most industries.}
\item{He watches TV so loud that I can't sleep.}
\end{enumerate}

\section{Verb Tenses}
If you see a date or time period in a sentence, it is probably a question about verb tense.  If there is no error in verb tense, then you can choose E) No error.

\bigskip
\subsection{A. Tense Consistency}
Generally, if a sentence starts in the present, it should stay in the present.  If it starts in the past, it should stay in the past.

\bigskip
Correct:  Since the student received no financial aid, she \longline to attend a community college for two years and then transfer.

\bigskip
Correct:  Since Bella hates snakes, she always \longline the reptile house at the zoo.

\subsection{B. Present Perfect vs. Simple Past}
Present perfect—has gone, has swum, has sung, has drunk

\bigskip
Simple past—went, swam, sang, drank.

\bigskip
Usually a sentence that includes a date or time period should have a verb in the simple past.  

\bigskip
Incorrect:  During the Salem Witch Trials in 1692, nineteen people have been accused of witchcraft and have hung on Gallows Hill.\\
Correct:  During the Salem Witch Trials in 1692, nineteen people \longline of witchcraft and hanged on Gallows Hill.

\bigskip
Incorrect:  During Queen Elizabeth's reign, Shakespeare has become a renowned playwright.
Correct:  During Queen Elizabeth's reign, Shakespeare \longline a renowned playwright.

\subsection{Would vs. Will}
Incorrect:  George Washington, who will become the first president of the United States, was born in 1732.\\
Correct:  George Washington, who \longline the first president of the United States, was born in 1732.

\bigskip
Do not use would or would have if the sentence or clause begins with if.
Incorrect:  If he would have arrived earlier, we would not have missed the movie. \\
Correct:  If he \longline arrived earlier, we would not have missed the movie.

\subsection{Gerunds vs. Infinitives}
Incorrect: Though he was one of the only students gaining admission to the Ivy League, Tom did not let his success go to his head.
Correct:  Though he was one of the only students to gain admission to the Ivy League, Tom did not let his success go to his head.

\bigskip
Incorrect:  The traffic prevented us to arrive in time.\\
Correct:  The traffic prevented us \longline in time.

\bigskip
Past perfect: \\
Incorrect:  By the time we started playing, the sun came out.\\
Correct:  By the time we started playing, the sun \longline out.

\subsection{Singular-Plural Noun Consistency}
Incorrect:  All the children wanted to be a superhero. \\
Correct:  All the children wanted to be \hrulefill 

\bigskip
Incorrect:  The boss gave her employees a handshake.\\
Correct:  The boss gave her employees \hrulefill

\bigskip
Incorrect:  Earth, water, air, and fire were considered an element by ancient scientists.\\
Correct:  Earth, water, air, and fire were considered  \longline by ancient scientists.

\section{Comparatives vs. Superlatives}
Comparatives, like better, worse, faster, smaller, compare two things or people.\\
Superlatives, like best, worst, fastest, and smallest, describe one thing as the best of many or compare three or more things or people.

\bigskip
Incorrect:  Between Jim and Bob, Jim was the best student and Bob was the worst student. \\
Correct:  Between Jim and Bob, Jim was the better student and Bob was the worse student.

\bigskip
Incorrect:  Texas is the biggest state in the continental U.S., and it has more oil.
Correct:  Texas is the biggest state in the continental U.S., and it has the \longline oil.

\section{Sentence Fragments}
Fragments may contain a noun and a verb but they do not contain one or more independent clauses. Fragments must be corrected to form complete sentences. 

\bigskip
A fragment may not be a sentence because...
\begin{enumerate}
\item{It describes something, but there is no subject-verb relationship}
\item{It may have most of the makings of a sentence but still be missing an important part of a verb string.}
\item{It may locate something in time and place with a prepositional phrase or a series of such phrases, but it's still lacking a proper subject-verb relationship within an independent clause.}
\item{It may even have a subject-verb relationship, but it has been subordinated to another idea by a dependent word and so cannot stand by itself.}
\end{enumerate}

\bigskip
\textbf{Practice: Turn the following sentence fragments into complete sentences.}
\begin{enumerate}
\item{In the middle of the day, when most people are at lunch, while I enjoy the quiet of my classroom.} \hrulefill 
\item{Although vanilla is the most popular ice cream flavor, but some people consider it boring.} \hrulefill
\end{enumerate} 

\section{Shift in Point of View}
The subject in each clause should be consistent.  For example, it is incorrect to switch from 2nd person “you” to third person “one.”

\bigskip
Incorrect: Even when I leave at 7 am, traffic gets me stuck.\\
Correct:  Even when I leave at 7 am, I get stuck in traffic.

\bigskip
Incorrect:  You should condition your hair if one doesn't want it to be dry.
Correct:  You should condition your hair if \longline don't want it to be dry.

\bigskip
Incorrect: If one does not study, you will not get a high score.
Correct:  If \longline  do not study, you will not get a high score.  OR  If \longline  does not study, one will not get a high score.

\subsection{You and Me Errors}
I = subject\\
Me = object

\bigskip
In an example with a plural subject, such as ``my friends and me," eliminate ``my friends" and check if the sentence still makes sense.  You cannot say ``Me went to the carnival" or ``Are you going to go white water rafting with I?"

\bigskip
Incorrect:  My friend and me went to the carnival.\\
Correct:  My friend and I went to the carnival.

\bigskip
Incorrect:  Are you going to go white water rafting with my friends and I? \\
Correct:  Are you going to go white water rafting with my friends and me?

\section{Idioms, Prepositions, and Commonly Confused Phrases}

\begin{spacing}{2}
\begin{tabular}{@{}ll<{\longline\arraybackslash}}
Capable&\\
Opposed&\\
Prohibited&\\
Comply&\\
Care&\\
Defined&\\
View&\\
Accompanied&\\
Benefit&\\
Contrary&\\
Oblivious&\\
Preoccupied&\\
Insist&\\
Recover&\\
Rely&\\
Subscribe&\\
Succeed&\\
Differ&\\
Discriminate&\\
Apply&
\end{tabular}
\end{spacing}

\textit{Look online for more examples!}

\section{Redundancy}
Watch out for wordiness and unnecessary repetition.

\bigskip
Incorrect:  The reason why the buffalo were endangered was because of over-hunting and habitat destruction.\\
Correct:  The buffalo were endangered, because of over-hunting and habitat destruction.

\bigskip
Incorrect:  According to the weather report, a blizzard was imminent in the future.\\
Fix it: \hrulefill

\section{Word Choice}
Watch out for easily confused words, like allusion and illusion or averse and adverse.  Allusion and illusion are called homophones, or words that sound the same but have different spellings and meanings.

\bigskip
Incorrect:  The magician was skilled in creating allusions.\\
Correct:  The magician was skilled in creating illusions.

\bigskip
Incorrect:  The ice storm caused averse road conditions.\\
Correct:  The ice storm caused \longline road conditions.

\section{Hypothetical Statements and the Subjunctive}
The subjunctive is used in sentences that express hypothetical situations, including a wish, emotion, possibility, judgment, opinion, necessity, or action that hasn't happened yet.

\bigskip
Incorrect:  If I was rich, I would travel the world.\\
Correct:  If I were rich, I would travel the world.

\bigskip
Incorrect: If she had showered earlier, she would not be late to dinner.\\
Correct:  If she had showered earlier, she would not \longline late to dinner.

\bigskip
Incorrect:  I would be happy if he was to call before dinner.\\
Correct:  I would be happy if he \longline to call before dinner.

\bigskip
Incorrect:  It is necessary that he arrives at 6:30.\\
Correct:  It is necessary that he \longline at 6:30.

\bigskip
Incorrect:  If Carla would have trained more, she would have finished the race.\\
Correct:  If Carla \longline trained more, she would have finished the race.

\section{That vs. Which}
That- Use that to introduce clauses with no commas, or restrictive clauses.

\bigskip
Dogs \longline don't wag their tails might try to bite you.

\bigskip
Which- Use which to introduce clauses with commas, or non-restrictive clauses.

\bigskip
Growling, \longline is usually a sign of aggression in dogs, is a warning to back off.

\vfill
\section{Among vs. Between}
Use between for two things, places, or people.\\
For example: Brian and Mike split the pizza \longline themselves.

\bigskip
Use among for three or more things, places, or people.\\
For example:  Brian, Mike, and Luigi split the pizza \longline  themselves.

\vfill
\section{Who versus Whom}
Who is used when the noun or pronoun is doing the action (it is a direct object like he), whereas whom is used when the noun or pronoun when an action is being done to it (it is an indirect action pronoun like him).

\bigskip
For example, ``Who is throwing the spiders at the children?" is correct and``To whom is Marla throwing spiders at?"

\bigskip
In this example, who is the direct object pronoun and whom is the indirect object pronoun.

\bigskip
\textbf{Fill in the following examples:}

\begin{enumerate}
\item{The woman, \longline I think is a slob, showed up to the interview wearing a Hawaiian shirt and jean shorts. }
\item{With \longline are you speaking?}
\end{enumerate}